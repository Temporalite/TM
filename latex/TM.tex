% -*-TeX root = TM.tex
%%%%%%%%%%%%%%%%%%%%%%%%%%%%%%%%%%%%%%%%%
% The Legrand Orange Book
% LaTeX Template
% Version 2.1.1 (14/2/16)
%
% This template has been downloaded from:
% http://www.LaTeXTemplates.com
%
% Original author:
% Mathias Legrand (legrand.mathias@gmail.com) with modifications by:
% Vel (vel@latextemplates.com)% 
%
% License:
% CC BY-NC-SA 3.0 (http://creativecommons.org/licenses/by-nc-sa/3.0/)
%
% Important note:
% Chapter heading images should have a 2:1 width:height ratio,
% e.g. 920px width and 460px height.
%
%%%%%%%%%%%%%%%%%%%%%%%%%%%%%%%%%%%%%%%%%

% LATEXING SHORTCUTS
% cmd + l : fill a blank (for references…)
% cmd + maj + b : build 
% cmd + ctrl + b : custom build option
% cmd + l, cmd + b : build option
% cmd + l, cmd + o : open the pdf
% cmd + l, cmd + d : online lookup 
% § : insert symbol
% cmd + l, cmd + n : create new environement
% cmd + l, cmd + s : show snippets
% cmd + l, cmd + w, b | e | u : make the selectioned text bold | italic | undeline
% tab : escape character
% @ : snippets for bibtex
%cmd + l, cmd + p : show all the commands

%----------------------------------------------------------------------------------------
%	PACKAGES AND OTHER DOCUMENT CONFIGURATIONS
%----------------------------------------------------------------------------------------

\documentclass[12pt,fleqn,oneside,openany]{book} % Default font size and left-justified equations

%----------------------------------------------------------------------------------------
%	VARIOUS REQUIRED PACKAGES
%----------------------------------------------------------------------------------------

\usepackage[top=3cm,bottom=3cm,left=3.2cm,right=3.2cm,headsep=10pt,a4paper]{geometry} % Page margins

\usepackage{xcolor} % Required for specifying colors by name
\definecolor{ocre}{RGB}{211, 84, 0} % couleur du doc

\usepackage{titlesec} % Allows customization of titles

\usepackage{graphicx} % Required for including pictures
\graphicspath{{img/}} % Specifies the directory where pictures are stored

\usepackage{lipsum} % Inserts dummy text

\usepackage{tikz} % Required for drawing custom shapes

\usepackage[francais]{babel} % Tout ce qu'il faut pour supporter un texte écrit en français

\usepackage{enumitem} % Customize lists
\setlist{nolistsep} % Reduce spacing between bullet points and numbered lists

\usepackage{booktabs} % Required for nicer horizontal rules in tables

\usepackage{eso-pic} % Required for specifying an image background in the title page

\usepackage{pdfpages} % pour les annexes

%----------------------------------------------------------------------------------------
%	POLICES
%----------------------------------------------------------------------------------------

% Font Settings
\usepackage{avant} % Use the Avantgarde font for headings
%\usepackage{times} % Use the Times font for headings
\usepackage{mathptmx} % Use the Adobe Times Roman as the default text font together with math symbols from the Sym­bol, Chancery and Com­puter Modern fonts

\usepackage{microtype} % Slightly tweak font spacing for aesthetics
\usepackage[utf8]{inputenc} % Required for including letters with accents
\usepackage[T1]{fontenc} % Use 8-bit encoding that has 256 glyphs

%----------------------------------------------------------------------------------------
%	MAIN TABLE OF CONTENTS
%----------------------------------------------------------------------------------------

\usepackage{titletoc} % Required for manipulating the table of contents

\contentsmargin{0cm} % Removes the default margin
% Chapter text styling
\titlecontents{chapter}[1.25cm] % Indentation
{\addvspace{15pt}\large\sffamily\bfseries} % Spacing and font options for chapters
{\color{ocre!60}\contentslabel[\Large\thecontentslabel]{1.25cm}\color{ocre}} % Chapter number
{}  
{\color{ocre!60}\normalsize\sffamily\bfseries\;\titlerule*[.5pc]{.}\;\thecontentspage} % Page number
% Section text styling
\titlecontents{section}[1.25cm] % Indentation
{\addvspace{5pt}\sffamily\bfseries} % Spacing and font options for sections
{\contentslabel[\thecontentslabel]{1.25cm}} % Section number
{}
{\sffamily\hfill\color{black}\thecontentspage} % Page number
[]
% Subsection text styling
\titlecontents{subsection}[1.25cm] % Indentation
{\addvspace{1pt}\sffamily\small} % Spacing and font options for subsections
{\contentslabel[\thecontentslabel]{1.25cm}} % Subsection number
{}
{\sffamily\;\titlerule*[.5pc]{.}\;\thecontentspage} % Page number
[] 

%----------------------------------------------------------------------------------------
%	MINI TABLE OF CONTENTS IN CHAPTER HEADS
%----------------------------------------------------------------------------------------

% Section text styling
\titlecontents{lsection}[0em] % Indendating
{\footnotesize\sffamily} % Font settings
{}
{}
{}

% Subsection text styling
\titlecontents{lsubsection}[.5em] % Indentation
{\normalfont\footnotesize\sffamily} % Font settings
{}
{}
{}
 
%----------------------------------------------------------------------------------------
%	PAGE HEADERS
%----------------------------------------------------------------------------------------

\usepackage{fancyhdr} % Required for header and footer configuration

\pagestyle{fancy}
\renewcommand{\chaptermark}[1]{\markboth{\sffamily\normalsize\bfseries\chaptername\ \thechapter.\ #1}{}} % Chapter text font settings
\renewcommand{\sectionmark}[1]{\markright{\sffamily\normalsize\thesection\hspace{5pt}#1}{}} % Section text font settings
\fancyhf{} \fancyhead[RO]{\sffamily\normalsize\thepage} % Font setting for the page number in the header
\fancyhead[L]{\leftmark} % Print the current chapter name on the left side of  page

\renewcommand{\headrulewidth}{0.5pt} % Width of the rule under the header
\addtolength{\headheight}{2.5pt} % Increase the spacing around the header slightly
\renewcommand{\footrulewidth}{0pt} % Removes the rule in the footer
\fancypagestyle{plain}{\fancyhead{}\renewcommand{\headrulewidth}{0pt}} % Style for when a plain pagestyle is specified

% Removes the header from odd empty pages at the end of chapters
\makeatletter
\renewcommand{\cleardoublepage}{
\clearpage\ifodd\c@page\else
\hbox{}
\vspace*{\fill}
\thispagestyle{empty}
\newpage
\fi}

%----------------------------------------------------------------------------------------
%	THEOREM STYLES
%----------------------------------------------------------------------------------------

\usepackage{amsmath,amsfonts,amssymb,amsthm} % For math equations, theorems, symbols, etc

\newcommand{\intoo}[2]{\mathopen{]}#1\,;#2\mathclose{[}}
\newcommand{\ud}{\mathop{\mathrm{{}d}}\mathopen{}}
\newcommand{\intff}[2]{\mathopen{[}#1\,;#2\mathclose{]}}
\newtheorem{notation}{Notation}[chapter]

%%%%%%%%%%%%%%%%%%%%%%%%%%%%%%%%%%%%%%%%%%%%%%%%%%%%%%%%%%%%%%%%%%%%%%%%%%%
%%%%%%%%%%%%%%%%%%%% dedicated to boxed/framed environements %%%%%%%%%%%%%%
%%%%%%%%%%%%%%%%%%%%%%%%%%%%%%%%%%%%%%%%%%%%%%%%%%%%%%%%%%%%%%%%%%%%%%%%%%%
\newtheoremstyle{ocrenumbox}% % Theorem style name
{0pt}% Space above
{0pt}% Space below
{\normalfont}% % Body font
{}% Indent amount
{\small\bf\sffamily\color{ocre}}% % Theorem head font
{\;}% Punctuation after theorem head
{0.25em}% Space after theorem head
{\small\sffamily\color{ocre}\thmname{#1}\nobreakspace\thmnumber{\@ifnotempty{#1}{}\@upn{#2}}% Theorem text (e.g. Theorem 2.1)
\thmnote{\nobreakspace\the\thm@notefont\sffamily\bfseries\color{black}---\nobreakspace#3.}} % Optional theorem note
\renewcommand{\qedsymbol}{$\blacksquare$}% Optional qed square

\newtheoremstyle{blacknumex}% Theorem style name
{5pt}% Space above
{5pt}% Space below
{\normalfont}% Body font
{} % Indent amount
{\small\bf\sffamily}% Theorem head font
{\;}% Punctuation after theorem head
{0.25em}% Space after theorem head
{\small\sffamily{\tiny\ensuremath{\blacksquare}}\nobreakspace\thmname{#1}\nobreakspace\thmnumber{\@ifnotempty{#1}{}\@upn{#2}}% Theorem text (e.g. Theorem 2.1)
\thmnote{\nobreakspace\the\thm@notefont\sffamily\bfseries---\nobreakspace#3.}}% Optional theorem note

\newtheoremstyle{blacknumbox} % Theorem style name
{0pt}% Space above
{0pt}% Space below
{\normalfont}% Body font
{}% Indent amount
{\small\bf\sffamily}% Theorem head font
{\;}% Punctuation after theorem head
{0.25em}% Space after theorem head
{\small\sffamily\thmname{#1}\nobreakspace\thmnumber{\@ifnotempty{#1}{}\@upn{#2}}% Theorem text (e.g. Theorem 2.1)
\thmnote{\nobreakspace\the\thm@notefont\sffamily\bfseries---\nobreakspace#3.}}% Optional theorem note

%%%%%%%%%%%%%%%%%%%%%%%%%%%%%%%%%%%%%%%%%%%%%%%%%%%%%%%%%%%%%%%%%%%%%%%%%%%
%%%%%%%%%%%%% dedicated to non-boxed/non-framed environements %%%%%%%%%%%%%
%%%%%%%%%%%%%%%%%%%%%%%%%%%%%%%%%%%%%%%%%%%%%%%%%%%%%%%%%%%%%%%%%%%%%%%%%%%
\newtheoremstyle{ocrenum}% % Theorem style name
{5pt}% Space above
{5pt}% Space below
{\normalfont}% % Body font
{}% Indent amount
{\small\bf\sffamily\color{ocre}}% % Theorem head font
{\;}% Punctuation after theorem head
{0.25em}% Space after theorem head
{\small\sffamily\color{ocre}\thmname{#1}\nobreakspace\thmnumber{\@ifnotempty{#1}{}\@upn{#2}}% Theorem text (e.g. Theorem 2.1)
\thmnote{\nobreakspace\the\thm@notefont\sffamily\bfseries\color{black}---\nobreakspace#3.}} % Optional theorem note
\renewcommand{\qedsymbol}{$\blacksquare$}% Optional qed square
\makeatother

% Defines the theorem text style for each type of theorem to one of the three styles above
\newcounter{dummy} 
\numberwithin{dummy}{section}
\theoremstyle{ocrenumbox}
\newtheorem{theoremeT}[dummy]{Theorem}
\newtheorem{problem}{Problem}[chapter]
\newtheorem{exerciseT}{Exercise}[chapter]
\theoremstyle{blacknumex}
\newtheorem{exampleT}{Example}[chapter]
\theoremstyle{blacknumbox}
\newtheorem{vocabulary}{Vocabulary}[chapter]
\newtheorem{definitionT}{Definition}[section]
\newtheorem{corollaryT}[dummy]{Corollary}
\theoremstyle{ocrenum}
\newtheorem{proposition}[dummy]{Proposition}

%----------------------------------------------------------------------------------------
%	DEFINITION OF COLORED BOXES
%----------------------------------------------------------------------------------------

\RequirePackage[framemethod=default]{mdframed} % Required for creating the theorem, definition, exercise and corollary boxes

% Theorem box
\newmdenv[skipabove=7pt,
skipbelow=7pt,
backgroundcolor=black!5,
linecolor=ocre,
innerleftmargin=5pt,
innerrightmargin=5pt,
innertopmargin=5pt,
leftmargin=0cm,
rightmargin=0cm,
innerbottommargin=5pt]{tBox}

% Exercise box	  
\newmdenv[skipabove=7pt,
skipbelow=7pt,
rightline=false,
leftline=true,
topline=false,
bottomline=false,
backgroundcolor=ocre!10,
linecolor=ocre,
innerleftmargin=5pt,
innerrightmargin=5pt,
innertopmargin=5pt,
innerbottommargin=5pt,
leftmargin=0cm,
rightmargin=0cm,
linewidth=4pt]{eBox}	

% Definition box
\newmdenv[skipabove=7pt,
skipbelow=7pt,
rightline=false,
leftline=true,
topline=false,
bottomline=false,
linecolor=ocre,
innerleftmargin=5pt,
innerrightmargin=5pt,
innertopmargin=0pt,
leftmargin=0cm,
rightmargin=0cm,
linewidth=4pt,
innerbottommargin=0pt]{dBox}	

% Corollary box
\newmdenv[skipabove=7pt,
skipbelow=7pt,
rightline=false,
leftline=true,
topline=false,
bottomline=false,
linecolor=gray,
backgroundcolor=black!5,
innerleftmargin=5pt,
innerrightmargin=5pt,
innertopmargin=5pt,
leftmargin=0cm,
rightmargin=0cm,
linewidth=4pt,
innerbottommargin=5pt]{cBox}

% Creates an environment for each type of theorem and assigns it a theorem text style from the "Theorem Styles" section above and a colored box from above
\newenvironment{theorem}{\begin{tBox}\begin{theoremeT}}{\end{theoremeT}\end{tBox}}
\newenvironment{exercise}{\begin{eBox}\begin{exerciseT}}{\hfill{\color{ocre}\tiny\ensuremath{\blacksquare}}\end{exerciseT}\end{eBox}}				  
\newenvironment{definition}{\begin{dBox}\begin{definitionT}}{\end{definitionT}\end{dBox}}	
\newenvironment{example}{\begin{exampleT}}{\hfill{\tiny\ensuremath{\blacksquare}}\end{exampleT}}		
\newenvironment{corollary}{\begin{cBox}\begin{corollaryT}}{\end{corollaryT}\end{cBox}}	

%----------------------------------------------------------------------------------------
%	REMARK ENVIRONMENT
%----------------------------------------------------------------------------------------

\newenvironment{remark}{\par\vspace{10pt}\small % Vertical white space above the remark and smaller font size
\begin{list}{}{
\leftmargin=35pt % Indentation on the left
\rightmargin=25pt}\item\ignorespaces % Indentation on the right
\makebox[-2.5pt]{\begin{tikzpicture}[overlay]
\node[draw=ocre!60,line width=1pt,circle,fill=ocre!25,font=\sffamily\bfseries,inner sep=2pt,outer sep=0pt] at (-15pt,0pt){\textcolor{ocre}{R}};\end{tikzpicture}} % Orange R in a circle
\advance\baselineskip -1pt}{\end{list}\vskip5pt} % Tighter line spacing and white space after remark

%----------------------------------------------------------------------------------------
%	SECTION NUMBERING IN THE MARGIN
%----------------------------------------------------------------------------------------

\makeatletter
\renewcommand{\@seccntformat}[1]{\llap{\textcolor{ocre}{\csname the#1\endcsname}\hspace{1em}}}                    
\renewcommand{\section}{\@startsection{section}{1}{\z@}
{-4ex \@plus -1ex \@minus -.4ex}
{1ex \@plus.2ex }
{\normalfont\large\sffamily\bfseries}}
\renewcommand{\subsection}{\@startsection {subsection}{2}{\z@}
{-3ex \@plus -0.1ex \@minus -.4ex}
{0.5ex \@plus.2ex }
{\normalfont\sffamily\bfseries}}
\renewcommand{\subsubsection}{\@startsection {subsubsection}{3}{\z@}
{-2ex \@plus -0.1ex \@minus -.2ex}
{.2ex \@plus.2ex }
{\normalfont\small\sffamily\bfseries}}                        
\renewcommand\paragraph{\@startsection{paragraph}{4}{\z@}
{-2ex \@plus-.2ex \@minus .2ex}
{.1ex}
{\normalfont\small\sffamily\bfseries}}

%----------------------------------------------------------------------------------------
%	CHAPTER HEADINGS
%----------------------------------------------------------------------------------------

% The set-up below should be (sadly) manually adapted to the overall margin page septup controlled by the geometry package loaded in the main.tex document. It is possible to implement below the dimensions used in the goemetry package (top,bottom,left,right)... TO BE DONE

\newcommand{\thechapterimage}{}
\newcommand{\chapterimage}[1]{\renewcommand{\thechapterimage}{#1}}

% Numbered chapters with mini tableofcontents
\def\thechapter{\arabic{chapter}}
\def\@makechapterhead#1{
\thispagestyle{empty}
{\centering \normalfont\sffamily
\ifnum \c@secnumdepth >\m@ne
\if@mainmatter
\startcontents
\begin{tikzpicture}[remember picture,overlay]
\node at (current page.north west)
{\begin{tikzpicture}[remember picture,overlay]
\node[anchor=north west,inner sep=0pt] at (0,0) {\includegraphics[width=\paperwidth]{\thechapterimage}};
%%%%%%%%%%%%%%%%%%%%%%%%%%%%%%%%%%%%%%%%%%%%%%%%%%%%%%%%%%%%%%%%%%%%%%%%%%%%%%%%%%%%%
% Commenting the 3 lines below removes the small contents box in the chapter heading
%\fill[color=ocre!10!white,opacity=.6] (1cm,0) rectangle (8cm,-7cm);
%\node[anchor=north west] at (1.1cm,.35cm) {\parbox[t][8cm][t]{6.5cm}{\huge\bfseries\flushleft \printcontents{l}{1}{\setcounter{tocdepth}{2}}}};
\draw[anchor=west] (3cm,-9cm) node [rounded corners=20pt,fill=ocre!10!white,text opacity=1,draw=ocre,draw opacity=1,line width=1.5pt,fill opacity=.6,inner sep=12pt]{\huge\sffamily\bfseries\textcolor{black}{\thechapter. #1\strut\makebox[22cm]{}}};
%%%%%%%%%%%%%%%%%%%%%%%%%%%%%%%%%%%%%%%%%%%%%%%%%%%%%%%%%%%%%%%%%%%%%%%%%%%%%%%%%%%%%
\end{tikzpicture}};
\end{tikzpicture}}
\par\vspace*{230\p@}
\fi
\fi}

% Unnumbered chapters without mini tableofcontents (could be added though) 
\def\@makeschapterhead#1{
\thispagestyle{empty}
{\centering \normalfont\sffamily
\ifnum \c@secnumdepth >\m@ne
\if@mainmatter
\begin{tikzpicture}[remember picture,overlay]
\node at (current page.north west)
{\begin{tikzpicture}[remember picture,overlay]
\node[anchor=north west,inner sep=0pt] at (0,0) {\includegraphics[width=\paperwidth]{\thechapterimage}};
\draw[anchor=west] (3cm,-9cm) node [rounded corners=20pt,fill=ocre!10!white,fill opacity=.6,inner sep=12pt,text opacity=1,draw=ocre,draw opacity=1,line width=1.5pt]{\huge\sffamily\bfseries\textcolor{black}{#1\strut\makebox[22cm]{}}};
\end{tikzpicture}};
\end{tikzpicture}}
\par\vspace*{230\p@}
\fi
\fi
}
\makeatother

%----------------------------------------------------------------------------------------
%	BIBLIOGRAPHIE
%----------------------------------------------------------------------------------------

% Bibliography
\usepackage[sorting=nyt,sortcites=true,autopunct=true,babel=hyphen,hyperref=true,abbreviate=false,backref=true,backend=biber]{biblatex} %
\addbibresource{bibliography.bib} % BibTeX bibliography file
\defbibheading{bibempty}{}

%----------------------------------------------------------------------------------------
%	HYPERLINKS IN THE DOCUMENTS
%----------------------------------------------------------------------------------------

% For an unclear reason, the package should be loaded now and not later
\usepackage{hyperref}
\hypersetup{hidelinks,backref=true,pagebackref=true,hyperindex=true,colorlinks=false,breaklinks=true,urlcolor= ocre,bookmarks=true,bookmarksopen=false,pdftitle={Title},pdfauthor={Author}}
 % Insert the commands.tex file which contains the majority of the structure behind the template

%----------------------------------------------------------------------------------------

\begin{document}

%----------------------------------------------------------------------------------------
%	TITLE PAGE
%----------------------------------------------------------------------------------------

\begingroup
\thispagestyle{empty}
\AddToShipoutPicture*{\put(0,0){\includegraphics[scale=1]{titre.jpg}}} % Image background
\centering
\vspace*{6,8cm}
\par\normalfont\fontsize{35}{35}\sffamily\selectfont

\textbf{La perception du temps}\\
{\LARGE Travail de maturité}\par % Book title
\vspace*{0,8cm}
\begin{minipage}{0.445\textwidth}
	\begin{flushleft} \large
		\emph{Auteur :}\\
		{\Large Lucas \textsc{Shooner}} % Your name
	\end{flushleft}
\end{minipage}
~
\begin{minipage}{0.445\textwidth}
	\begin{flushright} \large
		\emph{Superviseur :} \\
		{\Large Nicolas \textsc{Fiechter}} % Supervisor's Name
	\end{flushright}
\end{minipage} \\ 
{{\large \textsc{Gymnase du Bugnon site de Sévelin}}} \\
{\large \today}\\ \par
\endgroup

%----------------------------------------------------------------------------------------
%	COPYRIGHT PAGE
%----------------------------------------------------------------------------------------

\newpage
~\vfill
\thispagestyle{empty}

%\noindent Copyright \copyright\ 2014 Andrea Hidalgo\\ % Copyright notice
\noindent \textsc{Gymnase du Bugnon site de Sévelin}\\

\noindent \textsc{\href{https://temporalite.github.io}{temporalite.github.io}}\\ % URL

\noindent Le modèle \emph{The Legrand Orange Book} distribuée par \href{legrand.mathias@gmail.com}{Mathias Legrand}
 selon la license CC BY-NC-SA 3.0 \href{https://creativecommons.org/licenses/by-nc-sa/3.0/}{ (\texttt{https://creativecommons.org/licenses/by-nc-sa/3.0/})} a été utilisé et modifié pour la mise en page de ce travail. \\ % License information

\noindent \textit{Imprimé en octobre 2016} % Printing/edition date
\cite[text]{}
%----------------------------------------------------------------------------------------
%	Sommaire
%----------------------------------------------------------------------------------------

\chapterimage{head1.jpg} % Table of contents heading image

\pagestyle{empty} % No headers

\tableofcontents % Print the table of contents itself

%\cleardoublepage % Forces the first chapter to start on an odd page so it's on the right

\pagestyle{fancy} % Print headers again

%----------------------------------------------------------------------------------------
%	Théorie
%----------------------------------------------------------------------------------------

\chapterimage{head2.jpg} % Chapter heading image

\chapter{Introduction} \label{cha:introduction}
\section{Préambule}
Avant de commencer ce travail, voici quelques mots à propos de ce sujet. J'ai toujours été intéressé par les sciences, c'est d'ailleurs pour cela que j'ai choisi l'option biologie et chimie au gymnase; je voulais exécuter un travail comportant une partie pratique et l'approche expérimentale ainsi que le choix d'un travail de maturité portant sur la biologie me semblaient donc approprié avec mes centres d'intérêts. En plus de cela, le fait que le titre \textit{Neurosciences} soit évasif m'a intéressé car il me garantissait un grande liberté dans le choix du sujet à développer.

Quant à la décision d'étudier la temporalité et plus particulièrement la perception du temps, elle s'explique par le fait que je devais choisir un sujet pour lequel je puisse réaliser une expérience facilement réalisable et avec peu de moyens. Le gymnase du Bugnon ne possédant malheureusement pas d'IRMf, il m'a fallu m'orienter vers une expérience dont les résultats seraient identifiables d'un point de vue extérieur. 

\section{Méthodologie} \label{sec:methodologie}
Les connaissances acquises au gymnase n'étant pas suffisantes à la réalisation de ce travail, j'ai dû me documenter en empruntant des documents à la bibliothèque cantonale universitaire et celle du gymnase. J'ai aussi profité de ressources en ligne, en particulier l'encyclopédie \textit{Wikipédia} qui m'a permis de cerner mon sujet par son approche généraliste.

Ces premières ressources m'ont servi de base théorique pour la réalisation de ce travail mais une part importante de celui-ci se situe dans l'interprétation des résultats de mon expérience, j'ai alors profité de la méthode utilisée lors de la réalisation des travaux pratiques.
% DEVELOPER

Au niveau rédactionnel, il m'a semblé important de mettre un soin particulier dans la recherche des outils utilisés pour la réalisation de ce texte. C'est pour cela que j'ai cherché à bien faire les choses. J'ai ainsi appris à utiliser \LaTeX pour rédiger mon manuscrit; cela me garantissait une mise en page soignée et une grande aisance dans la gestion d'un long document, chose qui aurait été plus difficile à obtenir dans un logiciel de traitement de texte tel que Word. J'ai ensuite appris à utiliser Git, un logiciel de gestion de version, pour héberger mon travail sur la plateforme GitHub. Cela m'a permis de travailler sereinement en m'assurant que mon travail soit versionné et facilement récupérable en cas de perte.
% besoin de changer (meilleur workflow) -> besoin de nouveaux outils. Choix des outils (pour/contre, facilité, utilité). Apprentissage PARLER D?ANTIDOTE POUR LA CORRECTION ET LES SYNONYMES AINSI QUE POUR LE STYLE PARLER DE INKSCAPE POUR LES IMAGES

\section{Problématique} \label{sec:problematique}
Ma problématique se divise en deux parties auxquelles je répondrai dans un premiers temps de façon théorique, puis par la pratique :

\paragraph{Comment la temporalité est-elle perçue par les humains ?} Nous voyons avec nos yeux, sentons avec notre nez et écoutons avec nos oreilles. S'il parait évident que nous avons des organes dédiés à certains sens, il est cependant plus difficile de comprendre comment nous percevons le temps, car ce n'est pas une chose que l'on peut observer en tant que tel. Cette partie sera étudiée en préparation de l'expérience. 

\paragraph{Peut-on l'influencer ? Si c'est le cas, dans quelle mesure est-ce possible ? } Cette question constituera le cœur de mon expérience. Je pourrai y répondre en analysant les résultats obtenus. 


\section{Introduction aux neurosciences} \label{sec:introNeuro}
\begin{remark}
	Cette partie, réalisée avec la volontée de restituer le plus fidèlement les sources dont elle est issue, n'est qu'un assemblage de connaissance issues de diverses recherches et lectures, le présent travail ne se limite alors qu'à une synthèse de celles-ci. 
\end{remark}

\subsection[Définition]{Définition \cite{wikineuro}} \label{ssec:definition} % définition générale des neurosciences
Les neurosciences, domaine que l'on pourrait croire limité au cerveau par l'amalgame entre ce terme et le mot \emph{neurone}\footnote{La langue française regorge de locutions telles que ``s'échauffer les neurones'', ``avoir les neurones rouillés'', ``titiller les neurones'' pour lesquelles \emph{neurone} et \emph{cerveau} sont synonyme.}, visent à l'étude du système nerveux dans sa globalité; elles sont le sujet de plusieures disciplines qui travaillent à divers ordres de grandeur, de l'échelle moléculaire à l'étude d'un organe tel que le cerveau, jusqu'à l'ensemble d'un organisme. Les nombreuses disciplines qui se partagent ce champ de recherche permettent alors de l'aborder de différentes manières : l'étude des interactions synaptiques, la compréhension d'un comportement et le développement d'une interface homme-machine appartenant respectivement à la biochimie, aux sciences cognitives et à l'ingénierie. Ces différents angles d'approche se divisent ensuite en deux catégories distinctes mais complémentaires : une approche psychologique (cognition, psychologie…) et une démarche scientifique (biologie, chimie, mathématique…). Ils se réconcilient finalement selon deux approches opposées l'une à l'autre. La première, dénomée ascendante ou \emph{bottom-up}, a pour but de comprendre le fonctionnement des constituants les plus élémentaires du système nerveux pour ensuite saisir les mécanismes qui régissent les structures supérieures et ce de manière récursive jusqu'à ce que la totalité du système soit compris. La méthode descendante ou \emph{top-down} se définit comme une rétro ingéniérie du système nerveux en considérant un organisme dans sa globalité, dans l'intention de comprendre la nature et le fonctionnement des circuits neuronaux par une observation du comportement. Les flux neuronaux et leur traitement sont ainsi caractérisables selon l'une ou l'autre de ces méthodes; les influx sensoriels (tels que la sensation d'un objet sur la peau) et leur traitement relèvent d'une approche ascendante car ils déclenchent une cascade de réactions sans qu'un but précis ne régisse l'ensemble du raisonnement. Les fonctions exécutives, qui permettent de modifier le comportement en fonction du contexte dans lequel l'on se trouve, fonctionnent au contraire selon un mode descendant car elles sont caractérisées par un dessein supérieur (tel que la planifcation d'une action) qui dirige le raisonnement vers une action à partir de differentes sources d'information \cite{bottomupTopdown}\cite{foncExec}.

\begin{figure}[h]
	\centering{
		\resizebox{\textwidth}{!}{\includegraphics{img/infographie.pdf}}
		\caption{Organisation du champ des neurosciences \cite{infographie}}
		\label{fig:infographie}
	}
\end{figure}

Si les neurosciences sont en général associées aux mécanismes neurobiologiques et donc à la cognition, ce champ de recherche offre des applications dans des domaines aussi divers que l'éducation (pour faciliter l'apprentissage \cite{pedagogie}) ou la justice (des tribunaux américains ont recours à l'IRMf pour démontrer le lien entre des comportements délictueux et des lésions cérébrales bien que cette pratique fasse polémique \cite{justice}).

\subsection[La temporalité]{La temporalité \cite{reptemps}} \label{ssec:temporalite} % définition générale de comment ça marche, des moyens pour étudier la temporalité, puis ceux que je pourrais/vais employer
L'homme est capable de percevoir le temps de façon générale en se référant à ce qui a déjà été vécu, ce qui est vécu et ce qui le sera. Ces notions de passé, présent et futur étant de nature imprécise, l'homme à développé un système d'unités temporelles pour pouvoir ordonner le temps. En se basant sur des cycles réguliers tels que l'alternance entre le jour et la nuit ou le passage d'un électron d'un niveau énergétique à un autre, il est possible de définir le temps de manière objective. 

Au niveau cérébral, la perception du temps est liée à diverses facultés. En effet, il est indispensable d'emmagasiner, de conserver et de récupérer des souvenirs pour se référer au passé. Il faut ainsi être capable de prévoir ou d'imaginer ce qui va se passer dans l'avenir pour appréhender le futur. Sur une échelle temporelle plus courte, le cerveau est aussi capable de gérer les rythmes qui règlent la périodicité des processus biologiques et parvient en outre à percevoir des durées de l'ordre de la miliseconde lors d'activités motrices, musicales ou verbales. \cite{perceptionTemps}

Cette perception peut être déformée par les émotions, le temps semble par exemple s'écouler plus lentement lorsque l'on traverse une dépression \cite{emotiontemps}.

%----------------------------------------------------------------------------------------
%	Expérience
%----------------------------------------------------------------------------------------

\chapterimage{head1.jpg}

\chapter{\'Etude expérimentale} \label{cha:etudeExp}


\section{But} \label{sec:butExp}
La perception du temps étant subjective, ce chapitre vise à déterminer s'il est possible de l'influencer. Il y a bien évidemment de multiples pistes à explorer pour tenter cela et il ne s'agira pas de répondre à la problématique de façon exhaustive mais plutôt d'effectuer des expériences permettant d'y apporter des éléments de réponse. 

Pour ce faire, il est demandé à un groupe de personnes ne portant pas de montre d'effectuer une activité qu'ils prennent pour le sujet principal du test afin de ne pas influencer les résultats. Il est ensuite demandé à un autre groupe de répéter la même tâche dans des conditions différentes. Leur estimation du temps écoulé lors de chaque expérience est finalement récoltée individuellement par le biais d'un formulaire pour pouvoir ensuite la comparer à la durée réelle et ainsi déterminer si ces différentes conditions ont joué un rôle dans leur perception temporelle.

Il s'agira en l'occurrence de remplir un sudoku ou de dessiner sur la feuille lors de la projection d'un film tantôt intéressant, tantôt ennuyeux. 

% ORDRE matériel (sudoku et questionnaire enannexe) mérhode résultats

\section{Définition du protocole} \label{sec:DefinitionProto}

\subsection{Préparation} \label{ssec:preparation}
Il est nécessaire de préparer un certain nombre de choses avant l'expérimentation proprement dite :

\begin{description}
	\item[Environ trente sujets] Il faut un nombre assez important de participants pour pouvoir analyser leur réponses. Bien que le nombre minimum de sujets pour établir des statistiques lors d'une expérience soit de 8 par groupe\footnote{Selon M. [jesaisplussonnom], expert lors de la présntation d'un TM sur les souvenirs simulés}, il est préférable d'en avoir une trentaine au total. Il faut aussi tenir compte de la capacité des salles du gymanase qui atteint rarement plus de 25 places, auquel cas la totalité des tables sont occupées, mettant ainsi certains participants dans une position inconfortable pour voir le film. Le nombre de participants est ainsi fixé à 16 personnes par groupe, le double du minimum.
	\item[Une salle] Elle doit être assez grande pour accueillir tous les participants, être le plus neutre possible et se situer dans une partie calme du gymnase pour éviter toute distraction. Elle doit disposer d'un projecteur avec un système de sonorisation et les éventuelles horloges sont démontées avant le début de l'expérience.
	\item[Un sudoku] Il faut préparer un sudoku de faible difficulté pour que tous les participants soient sur un pied d'égalité. Il est distribué à tous les cobayes qui doivent le remplir pendant que le film est projeté. Son rôle principal est de faire croire aux participants que le but de l'expérience est de réaliser deux tâches en même temps, il est en effet impossible de concentrer son attention sur le film et le sudoku. Il sert ensuite à concentrer tous les participants sur le même problème, pour tenter de les induire dans un état d'esprit commun afin de normaliser les résultats. Il est cependant illusoire de penser que la totalité des participants apprécient le sudoku de la même façon. Pour éviter qu'un rejet de cette activité ne perturbe quelqu'un, il faut laisser le choix de dessiner sur la feuille.
	\item[Deux films] Il faut qu'ils soient de même durée mais de nature très différentes car ce seront eux qui influenceront les sujets. Le premier doit être intéressant tandis que le deuxième doit être ennuyeux. Ils doivent être courts, puisqu'il est difficile de motiver quelqu'un à participer si l'expérience est longue. Le choix de ces films doit en outre être judicieux, car c'est à travers eux que l'expérimentateur risque d'introduire des biais expérimentaux, s'il sélectionne des vidéos n'étant pas considérés comme intéressantes ou ennuyeuses par les participants. 

	La vidéo ennuyante est un extrait d'une vidéo dans laquelle des voitures roulent sous un pont. La caméra est fixe, il fait nuit et la musique d'ambiance est répétitive, au même titre que la vidéo. L'extrait utilisé pour la seconde partie de l'expérience est opposé au précédent car il met en scène une interview de Sylvain Durif de façon dynamique; l'entretient se déroule à plusieurs endroits et la caméra est mobile. Les questions et réponses apportent une dynamique opposée à la première vidéo, le côté saugrenu de l'intervenant étant l'aspact le plus important dans le fait qu'il paraisse drôle et interessant. 

	Il doivent de plus être visionnés sans que la barre d'avancement ne s'affiche à l'écran, pour ne pas révéler leur durée. Pour ce faire, il est possible d'utiliser le logiciel VLC qui permets de glisser sous l'écran la partie affichant les informations temporelles
	\item[Des formulaires] Ils doivent poser des questions en rapport avec la pseudo expérience. Il y en a un pour chaque vidéo, leur structure est similaire et ils ont en commun la question à laquelle il est demandé d'estimer le temps écoulé durant l'expérience. Le nombre de questions est limité pour laisser aux sujets le temps de se concentrer sur leur réponses dans le but d'éviter qu'ils ne le fassent machinalement.
\end{description}

\subsection{méthode} \label{ssec:methode}
\begin{enumerate}
	\item Chercher assez de personnes pour pouvoir effectuer l'expérience et leur expliquer qu'elle est assez courte, en restant évasif s'ils demandent sa durée. Déterminer deux groupes égaux de sujets pour la suite de l'expérimentation.
	\item Préparer la salle, démonter les horloges et s'assurer que le beamer ansi que le système audio soient allumés et en état de marche. Le film est ouvert en mode plein écran et la barre d'avancement est cachée.
	\item Faire entrer les sujets du premier groupe, leur distribuer à chacun une page de sudoku. Leur expliquer que l'expérience porte sur l'attention sans donner plus d'indications. 
	\item Commencer la lecture du premier film (ennuyant) et leur indiquer qu'il peuvent commencer leur sudoku. Il faut ensuite préciser qu'ils ne sont pas obligés de le faire et qu'ils peuvent dessiner à la place. 
	\item À la fin de la vidéo, préciser que l'expérience est terminée, distribuer le formulaire approprié et récupérer les sudoku.
	\item Récupérer les formulaires.
	\item Répéter les étapes précédentes avec le second groupe et la vidéo interessante.
	\item Jeter les sudoku.
\end{enumerate}

%PLUS PRECIS !

\section{Résultats} \label{sec:resultats}

\begin{table}[h]
	\centering
	\caption{Expérience ennuyante (5 min 20 s)} \label{tbl:expA}
	\begin{tabular}{@{}lc@{}}
		\toprule 
		\multicolumn{1}{@{}c@{}}{Nom} & Temps perçu {[}min{]}\\ \midrule
		\qquad Barbara M. & 5,0 \\
		\qquad Laura T. & 5,0 \\
		\qquad Emmanuelle R. & 5,0 \\
		\qquad Hiba K. & 5,0 \\
		\qquad Issa T. & 5,0 \\
		\qquad Ahmed N. & 5,4 \\
		\qquad Imad B. & 6,0 \\
		\qquad Sofia M. & 6,5 \\
		\qquad Aliénor D. & 8,0 \\
		\qquad Anonyme 1 & 8,0 \\
		\qquad Kelly D. S. & 8,0 \\
		\qquad Anonyme 2 & 8,6 \\
		\qquad Oana P. & 9,0 \\
		\qquad Camila C. & 10,0 \\
		\qquad Aurélien P. & 10,0 \\
		\qquad Lucas R. & 10,0 \\ \bottomrule
	\end{tabular}
\end{table}

\begin{table}[h]
	\centering
	\caption{Expérience intéressante (5 min 30 s)} \label{tbl:expB}
	\begin{tabular}{@{}lc@{}}
		\toprule
		\multicolumn{1}{@{}c@{}}{Nom} & Temps perçu {[}min{}] \\ \midrule
		\qquad Anonyme 3 & 2,5 \\
		\qquad Sajinth P. & 3,5 \\
		\qquad Luna M. & 4,0 \\
		\qquad Khaled A. & 4,0 \\
		\qquad Léa P. & 5,0 \\
		\qquad Amira N. & 5,0 \\
		\qquad Nadia Z. & 5,0 \\
		\qquad Léa & 5,0 \\
		\qquad Arnaud B. & 5,5 \\
		\qquad Sekatski & 6,0 \\
		\qquad Anonyme 3 & 6,0 \\
		\qquad Arnaud S. & 6,0 \\
		\qquad Louis M. & 7,0 \\
		\qquad Ronald V. & 10,0 \\
		\qquad Nils R. & 10,0 \\
		\qquad Nicolas R. & 12,0 \\ \bottomrule
	\end{tabular}
\end{table}

\begin{table}[h!]
	\centering
	\caption{Analyse des résultats} \label{tbl:analyse}
	\begin{tabular}{@{}lcc@{}}
		\toprule
		\multicolumn{3}{r@{}}{Expérience} \\ 
		& A & B \\ \midrule
		Temps perçu minimum {[}minutes{]} & 5,0 & 2,5 \\ 
		Temps perçu maximum {[}minutes{]} & 10,0 & 12,0 \\ 
		Temps perçu moyen {[}minutes{]} & 7,2 & 6,0 \\
		Différence absolue du temps perçu {[}minutes{]} & 1,9 & 0,5 \\ 
		Différence relative du temps perçu {[}\%{]} & 35,0 & 9,7 \\ \bottomrule
	\end{tabular}
\end{table}
\clearpage 

\section{Analyse} \label{sec:analyse}

\subsection{Analyse critique} \label{ssec:analyseCrit}
Avant d'analyser les résultats, il me parait utile de recenser les biais expérimentaux qui ont pu interférer avec cette expérience. Il est évidemment difficile de se prononcer à propos de leur impact sur les résultats mais il est néanmoins nécessaire de les répertorier pour considérer cette expérience de façon critique.

J'ai réalisé la première partie de l'expérience (vidéo ennuyante) avec ma classe lors d'un cours de biologie. Bien que je leur aie demandé s'il étaient d'accord d'y participer, certains élèves s'y sont peut-être sentis obligés et ont alors abordé l'expérience d'un point de vue négatif. Cela a peut être contribué au fait que le temps ait paru passer plus lentement.

Quant à la deuxième partie de l'expérience (vidéo intéressante), une personne m'a indiqué avoir déjà vu la vidéo et connaitre approximativement sa durée. Malgré le fait que je n'en aie projeté qu'un extrait, cela l'a possiblement aidé à définir la durée de l'expérience.
Je l'ai en outre réalisée à la pause de midi, environ 15 minutes avant la reprise des cours. Cela a alors indiqué le temps maximal de l'expérience, à partir duquel il a été possible de déduire sa durée réelle.

De façon générale, le sodoku a été choisi comme tâche à réaliser lors l'expérience pour que tout les participants soient dans un même état d'esprit. Il permet ainsi d'unifier le référentiel temporel des participants par une activité ludique. Il est cependant probable que certains d'entre-eux n'en soient pas adeptes, causant ainsi une modification de leur état d'esprit par rapport au reste du groupe.
Ensuite, il est difficile de tirer des conclusions à partir d'échantillons aussi petits. Cette expérience dépend de divers facteurs et devrait être effectuée sur un échantillon plus important pour assurer la puissance statistique de ses résultats.
Il est aussi possible que la durée des vidéos ait été trop courte pour que les participants réfléchissent à sa durée. Il est probable que certains d'entre-eux aient considéré que le temps était court et aient répondu 5 minutes de la même manière que l'on répond « J'arrive dans 5 minutes. » pour signifier que l'on arrive bientôt sans pour autant être précis. 

% Meme si l'expérience est complexe -> pas le plus simple pour tester mon sujet : on peut imaginer que les sujets réagiseent de manière similaire entre entre les deux (difficulté à se concentrer, pas interessés par le sudoku ou trop interessé) -> les biais sont similaires -> compensation. 

% Personne n'a dessiné. Donc homogénéité des participants mais liberté -> pas de pression

% Il faut isoler la variable étudiée ! Difficile okay mais expliquer comment j'ai fait. Parler de la taille du groupe

\subsection{Analyse des résultats} \label{ssec:analyseResult}
Il est intéressant de constater qu'en moyenne, les temps perçus sont supérieurs à la durée réelle de l'expérience. Il y a cependant une différence entre les deux; il apparait que le temps a semblé passer plus lentement lorsque la vidéo ennuyante a été projetée. Cette dernière à en effet causé une différence de 35 \% par rapport au temps de référence. La vidéo intéressante provoque quant à elle une augmentation de 9,7 \% du temps perçu par rapport à sa durée réelle.

Il y donc une différence de 25,3 \% entre les deux temps perçus. Nous avons précédemment discuté des biais pouvant affecter ce résultat, il est donc à considérer avec prudence. 

\newpage
\section{Réponse à la problématique} \label{sec:reponseProb}

\newpage
\section{Conclusion} \label{sec:conclusion}
% conclusion du travail

% remerciements

%----------------------------------------------------------------------------------------
%	Annexes
%----------------------------------------------------------------------------------------

\chapterimage{head2.jpg}

\appendix
\chapter{Annexes} \label{cha:annexes}
\section*{Expérience 1}
\addcontentsline{toc}{section}{Expérience 1}

\subsection*{Sudoku} \label{sec:Sudoku}
\addcontentsline{toc}{subsection}{Sudoku}
\begin{figure}[htp] \centering{
		\includegraphics[trim=40 360 40 75, clip, width=\textwidth]{exp/Sudoku}} % gauche bas droite haut
	\caption{Sudoku distribué lors de l'expérience}
\end{figure}  

\newpage
\subsection*{Questionnaires} \label{sec:Questionnaires}
\addcontentsline{toc}{subsection}{Questionnaires}
\begin{figure}[htp] \centering{
		\includegraphics[trim=40 310 40 0,clip, width=\textwidth]{exp/QuestA}}
	\caption{Questionnaire distribué lors de la première expérience (A)}
\end{figure}

\newpage
\begin{figure}[htp] \centering{
		\includegraphics[trim= 40 274.6 40 0,clip, width=\textwidth]{exp/QuestB}}
	\caption{Questionnaire distribué lors de la première expérience (B)}
\end{figure}

\newpage
\section*{Expérience 2}
\addcontentsline{toc}{section}{Expérience 2}

%----------------------------------------------------------------------------------------
%	Bibliographie
%----------------------------------------------------------------------------------------

\chapterimage{head1.jpg} % Chapter heading image

\chapter{Bibliographie}
\begin{remark}
	Références non présentes dans le texte mais qu'il faudra bien placer un jour : \cite{ref1} \cite{vidnul} \cite{vidcool} \cite{imgtitre} \cite{imgheader1} \cite{analysedonnees}
\end{remark}
\section*{Livres}
\addcontentsline{toc}{section}{Livres}
\printbibliography[heading=bibempty,type=book]
\section*{Articles}
\addcontentsline{toc}{section}{Articles}
\printbibliography[heading=bibempty,type=article]
\section*{Ressources en ligne}
\addcontentsline{toc}{section}{Ressources en ligne}
\printbibliography[heading=bibempty,type=online]

%----------------------------------------------------------------------------------------

\end{document}